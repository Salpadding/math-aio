\section{Metric Space}

\subsection{Topological Space}

\begin{definition}[Topological Space]
\leavevmode
\begin{enumerate}
    \item A collection $\tau$ of subsets of a set $X$ is said to be a \emph{topology} on $X$ if $\tau$ has the following three properties:
    \begin{enumerate}
        \item $\varnothing \in \tau$ and $X \in \tau$.
        \item If $V_i \in \tau$ for $i = 1, \dots, n$, then $V_1 \cap V_2 \cap \cdots \cap V_n \in \tau$.
        \item If $\{ V_\alpha \}$ is an arbitrary collection of members of $\tau$ (finite, countable, or uncountable), then $\bigcup_\alpha V_\alpha \in \tau$.
    \end{enumerate}

    \item If $\tau$ is a topology on $X$, then $X$ is called a \emph{topological space}, and the members of $\tau$ are called the \emph{open sets} in $X$.
\end{enumerate}
\end{definition}

\subsubsection{Interior}

\begin{definition}[Interior Point]\label{fc673441}
    Let $(X, \tau)$ be a topological space and $E \subseteq X$. A point $p$ is an interior point of $E$ if and only if there exists a neighborhood $V$ of $p$ such that $V \subseteq E$.
\end{definition}


\begin{definition}[Interior]
    Let $(X, \tau)$ be a topological space and $E \subseteq X$. The interior of $E$ is the set of all interior points of $E$, often
    denoted as $\mathring{E}$ or $\mathrm{int}(E)$.

    \[
\mathring{E} = \{ p : p \text{ is an interior point of } E \}
    \]

\end{definition}

\begin{corollary}\label{bfb0d46a}
    The interior operator is monotone: if $A \subseteq B$, then $\mathring{A} \subseteq \mathring{B}$.
\end{corollary}


\begin{proof}
    Easily to prove by transitivity of $\subseteq$
\end{proof}

\begin{corollary}\label{f486ca55}
    For any set $E$, $\mathring{E} \subseteq E$
\end{corollary}

\begin{proof}
    By definition, every interior point of $A$ is contained in $A$, so $\mathring{A} \subseteq A$.
\end{proof}

\begin{thm}\label{cc4ec32c}
    The interior of any set is open.
\end{thm}

\begin{proof}
    Let $\mathring{A}$ denote the interior of $A$.
    Pick any point $p \in \mathring{A}$. 
    Since $p$ is an interior point, 
    there exists a neighborhood $V$ of $p$ such that $V \subseteq A$. 
    Now, for any $q \in V$, $q$ is also an interior point of $A$ because $V$ is a neighborhood of $q$ contained in $A$.

    \[
    \mathring{A} = \bigcup_{p \in A} \{ V_p : V \text{ open},\, V \subseteq \mathring{A} \}
    \]
    
    is a union of open sets, and hence is open.
\end{proof}

\begin{thm}
    For any set $A$, $A$ is open if and only if $\mathring{A} = A$.
\end{thm}

\begin{proof}
    If $A$ is open, any $p \in A$ is an interior point, 
    because $A$ is a neighborhood of $p$,
    which shows $A \subseteq \mathring{A}$, so $A = \mathring{A}$ by \cref{f486ca55}.

    Conversely, if $\mathring{A} = A$, then $A$ is open by \cref{cc4ec32c}.
\end{proof}

\begin{thm}
    For any $E\subseteq X$, the interior $\mathring{E}$ is the largest open subset of $E$; equivalently,
    \[
        \mathring{E} = \bigcup\{V \subseteq E : V \text{ is open} \} .
    \]
\end{thm}

\begin{proof}
    Let
    \[
        \mathcal{I} := \{V \subseteq E : V \text{ is open} \}, \qquad
        V' := \bigcup_{V \in \mathcal{I}} V .
    \]
    Then $V'$ is open (a union of open sets) and $V'\subseteq E$ by construction.

    \begin{enumerate}
        \item[($\subseteq$)] Since $\mathring{E}$ is open and $\mathring{E}\subseteq E$, 
        we have $\mathring{E}\in\mathcal{I}$, hence $\mathring{E}\subseteq V'$.

        \item[($\supseteq$)] If $p\in V'$, then $p\in V$ for some open $V\subseteq E$. 
        By the definition of interior point (\cref{fc673441}), this implies $p\in\mathring{E}$.
    \end{enumerate}

    Therefore $V'=\mathring{E}$, and $\mathring{E}$ is the largest open subset of $E$.
\end{proof}

\begin{thm}
    $\mathrm{int}(A \cap B) = \mathrm{int}(A) \cap \mathrm{int}(B)$
\end{thm}

\begin{proof}
    \:

   \begin{enumerate}
    \item [($\subseteq$)] By $\mathring{A}$ is monotone increasing function over $A$.
    
    \begin{align}
        A \cap B & \subseteq A \\
        \mathrm{int}(A \cap B) & \subseteq \mathrm{int}(A) \label{6919e47e} \\
        \mathrm{int}(A \cap B) & \subseteq \mathrm{int}(B) \label{baa9c1c7}
    \end{align}

    And apply $\cap$ at both side of \eqref{6919e47e} and \eqref{baa9c1c7}

    \item [($\supseteq$)]

    Define

    \begin{align*}
        I_A &= \{V \subseteq A: V \text{ is open} \} \\
        I_B &= \{U \subseteq B: U \text{ is open} \} \\
    \end{align*}

    Thus

    \begin{align*}
        \mathrm{int}(A) \cap \mathrm{int}(B) &= \bigcup_{V \in I_A} V \cap  \bigcup_{U \in I_B} U \\
        &=  \bigcup_{V \in I_A} \bigcup_{U \in I_B} V \cap U
    \end{align*}

    Since $V, U$ is open and $V \cap U \subseteq A \cap B$ for any $V \in I_A, U \in I_B$. $\mathrm{int}(A) \cap \mathrm{int}(B)$
    is union of open sets contained by $A \cap B$, and hence $\mathrm{int}(A) \cap \mathrm{int}(B) \subseteq \mathrm{int}(A \cap B)$
   \end{enumerate} 
\end{proof}

\begin{corollary}
    $\mathrm{int}(A) \cup \mathrm{int}(B) \subseteq \mathrm{int}(A \cup B)$
\end{corollary}

\subsubsection{Closure}

\begin{definition}[Closure]
    For a subset $A$ of a topological space $X$, the \emph{closure} of $A$ is defined by
    \[
        \overline{A} := \bigl(\mathrm{int}(A^C)\bigr)^C,
    \]
    where $A^C$ denotes the complement of $A$. This definition is convenient because
    many properties of closure follow directly from the corresponding properties of the interior.
\end{definition}

\begin{definition}[Boundary]
    The boundary of set $E$ is defined as $\overline{E} \setminus \mathrm{int}(E)$,
    often denoted as $\partial E$. 
\end{definition}

\begin{thm}
    For any $A\subseteq X$, the closure $\overline{A}$ satisfies:
    \begin{enumerate}
        \item $\overline{A}$ is closed, since $\mathrm{int}(A^C)$ is open.
        \item $A \subseteq \overline{A}$, because $\mathrm{int}(A^C)\subseteq A^C$ implies
        \[
            A \subseteq \bigl(\mathrm{int}(A^C)\bigr)^C = \overline{A}.
        \]
        \item If $A\subseteq B$, then $\overline{A}\subseteq \overline{B}$, since the interior operator is monotone.
    \end{enumerate}
\end{thm}


\begin{thm}
    $A$ is closed iff $\overline{A} = A$
\end{thm}

\begin{proof}
    Since $A^C$ is open iff $\mathrm{int}(A^C) = A^C$
\end{proof}

\begin{thm}
    $\overline{A}$ is minimum closed set which covers $A$
\end{thm}

\begin{proof}
    Consider that

    \[
        \{ K:  A \subseteq K,\: K \text{ is closed }\} = \{ V^C: V \subseteq A^C,\: V \text{ is open }\}
    \]

    Thus

    \begin{align*}
        \bigcap \{ K:  A \subseteq K,\: K \text{ is closed }\} &= \bigcap \{ V^C: V \subseteq A^C,\: V \text{ is open }\} \\
        &=  \left( \bigcup \{ V: V \subseteq A^C,\: V \text{ is open }\} \right)^C \\
        &= \bigl( \mathrm{int}(A^C) \bigr)^C = \overline{A}
    \end{align*}
\end{proof}

\begin{thm}
    $\overline{A \cup B} = \overline{A} \cup \overline{B}$ and  $\overline{A \cap B} \subseteq \overline{A} \cap \overline{B}$
\end{thm}

\begin{proof}
   By:
   
   \begin{align*}
    \overline{A \cup B} &= \bigl( \mathrm{int}(A^C \cap B^C) \bigr)^C = \bigl( \mathrm{int}(A^C) \cap \mathrm{int}(B^C) \bigr)^C \\
    & = \bigl( \mathrm{int}(A^C) \bigr)^C \cup \bigl( \mathrm{int}(B^C) \bigr)^C = \overline{A} \cup \overline{B}
   \end{align*}

   and


   \begin{align*}
    \overline{A \cap B} &= \bigl( \mathrm{int}(A^C \cup B^C) \bigr)^C \subseteq \bigl( \mathrm{int}(A^C) \cup \mathrm{int}(B^C) \bigr)^C \\
    & = \bigl( \mathrm{int}(A^C) \bigr)^C \cap \bigl( \mathrm{int}(B^C) \bigr)^C = \overline{A} \cap \overline{B}
   \end{align*}
\end{proof}

\begin{thm}
    $\overline{A} = A' \cup A$. $\overline{A}$ is the union of $A$ and its set of limit points.
\end{thm}

\begin{proof}
    First, we show that $A \cup A'$ is closed. Suppose $p \notin A \cup A'$. 
    Then $p \notin A$ and $p \notin A'$. 
    The latter means there exists a neighborhood $V$ of $p$ such that 
    $(V \setminus \{p\}) \cap A = \varnothing$. Since also $p \notin A$, we get $V \cap A = \varnothing$. 
    Hence $V \subseteq (A \cup A')^C$, showing $(A \cup A')^C$ is open and therefore $A \cup A'$ is closed.

    Clearly $\overline{A} \subseteq A \cup A'$, so by minimality of closure,
    \[
        \overline{A} \subseteq A \cup A'.
    \]

    Let's prove $\mathrm{int}(A^C) \subseteq (A')^C$. It is obviously that 
    $\mathrm{int}(A^C)$ is a neighborhood of all its elements who has empty
    intersection with $A$, which shows $p \notin A'$. Apply complement at both side,
    we got $A' \subseteq \overline{A}$, combine $A \subseteq \overline{A}$, we got 
    $A' \cup A \subseteq \overline{A}$
\end{proof}

\subsection{T1 Space}

\begin{thm}\label{03f30576}
    In T1 space, any single point set is closed.
\end{thm}

\begin{proof}
    Fix single point set $ E = \{ p \}$, consider $q \in E^C$, by $q \ne p$ and T1 axiom, there exists neighborhood $V_q$
    contains $q$ such that $p \notin V_q$. Thus $E^C$ is union of open sets, and hence open.

    \[
        E^C = \bigcup_{q \in E^C} V_q \quad  p \notin V_q
    \]
\end{proof}

\begin{thm}
    In a T1 space, the derived set (the set of all limit points) of any subset is closed.
\end{thm}

\begin{proof}
    Assume $E'$ is derive set of $E$, if $p \notin E'$, there exists a neighborhood $V$ of $p$ such that 

    \[
        V \cap E \subseteq \{ p \}
    \]

    Consider $U = V \setminus \{ p \}$; $U$ is open because $\{ p \}$ is closed in a T1 space(by \cref{03f30576}). And for any $q \in U$, $q$ is not a limit point of $E$.
\end{proof}

\begin{thm}\label{2dc4936a}
    In a T1 space, let $p$ be a limit point of $E$. For every neighborhood $V$ of $p$, $V \cap E$ is infinite.
\end{thm}

\begin{proof}
    Without loss of generality, we may assume $p \notin E$(by \cref{57d9c0d2}), since removing $p$ from $E$ does not affect whether $V \cap E$ is infinite. 
    A superset of an infinite set must also be infinite. 

    Suppose $p$ is a limit point of $E$. 
    Suppose, for the sake of contradiction, that $V \cap E = \{x_1, x_2, \dots, x_n\}$ is finite, where $V$ is a neighborhood of $p$.
    For each $x_k \in V \cap E$, the T1 axiom provides a neighborhood $U_k$ of $p$ such that $x_k \notin U_k$. 
    Define
    \[
        U = \bigcap_{k=1}^n U_k.
    \]
    Then $U$ is a neighborhood of $p$ (as a finite intersection of neighborhoods), and by construction, $U$ excludes all points $x_k \in V \cap E$, so $U \cap (V \cap E) = \emptyset$. 
    Thus, $U \cap V$ is a neighborhood of $p$ with $U \cap V \cap E = \emptyset$, contradicting the assumption that $p$ is a limit point of $E$.
\end{proof}


\begin{corollary}
    In a T1 space, a finite set has no limit points. 
\end{corollary}

\begin{proof}
    By \cref{2dc4936a}
\end{proof}

\begin{corollary}
    In a T1 space, if $E$ has a limit point $p$
\end{corollary}

\begin{proof}
    By \cref{2dc4936a}
\end{proof}

\subsection{First Countable}

\begin{definition}[Limit Point]
   A limit point $p$ of $E$ is a point such that for all open neighborhoods 
   $V$ containing $p$, we have $V \setminus \{ p \} \cap E \ne \emptyset$.
\end{definition}


\begin{corollary}\label{57d9c0d2}
    $p$ is limit point of $E$ iff $p$ is  limit point of $E \setminus \{ p \}$
\end{corollary}

\begin{proof}
    It is obviously by 

    \[
        \left(V \setminus \{ p \} \right) \cap E = \left(V \setminus \{ p \} \right) \cap \left( E \setminus \{ p \} \right)
    \]
\end{proof}

\begin{thm}\label{4ebc8233}
    In a first countable space $X$, a point $p$ is a limit point of $E\subset X$ iff 
    there exists a sequence $\{ x_n \}$ with $x_n\in E\setminus\{p\}$ and $x_n\to p$. Moreover, if $X$ is $T_1$, the sequence can be chosen with pairwise distinct terms. In particular, in a $T_1$ space one may remove finitely many points from $E$ without affecting whether $p$ is a limit point.
\end{thm}

\begin{proof}
    \begin{enumerate}
        \item $(\Rightarrow)$  
        
        Suppose 
        $x_n\in E\setminus\{p\}$ and $x_n\to p$. Then for every neighborhood $V$ of $p$,
        $x_n \in V$ for all but finitely $n$.
        Since $x_n\neq p$, we have $(V\setminus\{p\})\cap E\neq\varnothing$, so $p$ is a limit point of $E$.

        \item $(\Leftarrow)$ 
        
        Assume $X$ is first countable and $p$ is a limit point of $E$. 
        Let $(V_n)_{n\ge1}$ be a countable local base at $p$. 
        Define a nested base by $U_1:=V_1$ and $U_n:=\bigcap_{k=1}^n V_k$ for $n\ge2$. 
        By the definition of limit point, for each $n$ there exists $x_n\in (U_n\cap E)\setminus\{p\}$. 
        If $V$ is any neighborhood of $p$, $U_k \subseteq V, x_k \in V$ for all but finitely $k$, and therefore $x_n\to p$.

        \item Distinctness in $T_1$ 

        Now assume in addition that $X$ is $T_1$. 
        $U_n\cap E$ is infinite for every $n$ by \cref{2dc4936a}
        Hence each $U_n\cap E$ is infinite. 
        We can then choose inductively a sequence of distinct points $y_n\in (U_n\cap E)\setminus\{p,y_1,\dots,y_{n-1}\}$. 
        By the nesting argument above, $y_n\to p$.
    \end{enumerate}
\end{proof}


\begin{thm}
    If $X$ is first countable and T1 space, then for any set $E_1, E_2$ under $X$, $\big( E_1 \cup E_2 \big)' =  E_1' \cup E_2'$
\end{thm}


\begin{proof}
    Pick $x' \in \big( E_1 \cup E_2 \big)'$, by \cref{4ebc8233}, there exists distinct $x_n \in E_1 \cup E_2$
    and $x_n \to x'$.

    Consider set

    \begin{align*}
        W_1 &= \{ n \in \mathbb{N}: x_n \in E_1 \} \\
        W_2 &= \{ n \in \mathbb{N}: x_n \in E_2 \} \\
    \end{align*}

    At least one of $W_1$ or $W_2$ is infinite, and hence $x_n$ has a subsequence
    in $E_1$ or $E_2$
    converges to $x'$, which indicates $x' \in E_1' \cup E_2'$
\end{proof}

\subsection{Compact Set}
\begin{thm}
    In Hausdorff space, compact set is closed.
\end{thm}

\begin{proof}
   Assume $K$ is compact and $p \notin K$, we can construct an open cover of $K$.

   \[
        K \subseteq \bigcup_{q \in K} U_q \quad \text{where } U_q \cap V_q = \emptyset
   \]

   $U_q$ and $V_p$ is open neighborhood of $q,p$ respectively, thus we could pick a finite sub cover.

   \[
        K \subseteq \bigcup_{q \in I} U_q \quad I \subseteq K \quad I \text{ is finite}
   \]

   And define

   \[
    U = \bigcup_{q \in I} U_q \quad V = \bigcap_{q \in I} V_q
   \]

   Then we got $V \cap K \subseteq V \cap U \subseteq \emptyset$, which shows $K^C$ is open and $K$ is closed.
\end{proof}

\begin{thm}
    Compact set under metric space is bounded.
\end{thm}

\begin{proof}
    Consider a compact set $K$, and its open cover by infinite open balls.

    \[
       K \subseteq \bigcup_{q \in K} B(q, 1) 
    \]

    Thus there exists a finite open cover:

    \[
       K \subseteq \bigcup_{q \in I} B(q, 1) \quad I \subseteq K \quad I \text{ is finite}
    \]

    Pick $q_0 \in I$ and 

    \[
        r = \max_{q \in I} d(q_0, q)
    \]

    For any $p \in K$, we got $d(p, q_0) \le d(p, q) + d(q, q_0)$, where $p \in B(q, 1)$, and $d(p, q_0) \le 1 + r$
\end{proof}

\begin{thm}\label{e603cd67}
    Under a metric space, if $E \subseteq K$, where $E$ is infinite and $K$ is compact, then $E$ has at least
    one limit point in $K$
\end{thm}

\begin{proof}
    Assume $\forall p \in K, \exists V_p$ such that $V_p \cap E \subseteq \{ p \}$. Thus we got
    an open cover of $K$:

    \[
        K \subseteq \bigcup_{p \in K} V_p \quad V_p \cap E \subseteq \{p\}
    \]

    And hence we got a finite sub cover $I \subseteq K$ and 


    \[
        K \subseteq \bigcup_{p \in I} V_p \quad V_p \cap E \subseteq \{p\}
    \]

    Thus

    \[
        K \cap E \subseteq \bigcup_{p \in I} E \cap V_p \subseteq I
    \]

    which is contradict with $E$ is infinite.
\end{proof}

\begin{definition}[Sequentially Compact]
    A topological space is sequentially compact if every sequence has a convergent subsequence.
\end{definition}

\begin{thm}
    For a metric space $(X,d)$, it is compact iff it is sequentially compact.
\end{thm}

\begin{proof}
    \begin{enumerate}
        \item[($\Rightarrow$)]
        If $X$ is compact, consider $x_n$ be any sequence, if set $\{ x_n \}$ is finite, at least 
    one of $\{ x_n \}$ occurs infinitely many times, and hence a convergent subsequence. 
    If $\{ x_n \}$ is infinite, it must have a limit point in $X$, assume $x_0$ is a limit point of $\{ x_n \}$.

    Now we can pick $x_{n_1}$ such that $d(x_{n_1}, x_0) < \frac{1}{2}$, and $n_2 > n_1$ such that 
    $d(x_{n_2}, x_0) < \frac{1}{2^2}$, by \cref{2dc4936a} and thus construct a subsequence converges to $x_0$

        \item[($\Leftarrow$)]
Let $\{V_\alpha\}_{\alpha \in I}$ be an open cover of $X$. Define, for each $x \in X$,

\[
  r(x) := \sup\{\, r > 0 : \exists\, \alpha \in I \text{ with } B(x,r) \subseteq V_\alpha \,\}.
\]


We first show that $r$ is lower semicontinuous. Fix $a \in \mathbb{R}$. 
If $r(x) > a$,
then by definition there exist $\alpha$ and $r>a$ with $B(x,r) \subseteq V_\alpha$.
Let $\varepsilon = (r-a)/4 > 0$. For any $p \in B(x,\varepsilon)$ we have
$B(p,r-\varepsilon) \subseteq B(x,r) \subseteq V_\alpha$; since $r-\varepsilon > a+\varepsilon$,
it follows that $r(p) > a$. Hence the set $\{x \in X : r(x) > a\}$ is open, i.e. $r$ is
lower semicontinuous. In particular, for any sequence $x_n \to x$ in $E$,

\[
  \varliminf_{n \to \infty} r(x_n) \ge r(x).
\]

Set $r_0 := \inf\{ r(x) : x \in E \}$.

\textbf{Claim 1}: $r_0 > 0$. Suppose to the contrary that $r_0 = 0$. Then there exists a
sequence $x_n \in E$ with $r(x_n) \to 0$. Since $E$ is bounded, by \cref{metric-space-bolzano-weierstrass}
there is a subsequence $y_n = x_{f(n)}$ converging to some $y \in \mathbb{R}^n$; because
$E$ is closed, $y \in E$. As $\{V_\alpha\}$ covers $X$ and $y \in X$, there exists some
$\alpha_0$ with $y \in V_{\alpha_0}$; since $V_{\alpha_0}$ is open, there exists
$\delta>0$ with $B(y,\delta) \subseteq V_{\alpha_0}$. Hence $r(y) \ge \delta > 0$.
But lower semicontinuity gives
$
  \liminf_{n \to \infty} r(y_n) \ge r(y) > 0,
$
contradicting $r(y_n)=r(x_{f(n)}) \to 0$. Thus $r_0>0$.

\textbf{Claim 2}: $\{V_\alpha\}$ has a finite sub cover. For sake of contradiction, choose any $x_0 \in X$.
By definition of $r_0$ and the cover, there exists $\alpha_0$ with
$B(x_0,r_0) \subseteq V_{\alpha_0}$. Since $X$ not covered by $V_{\alpha_0}$,
we can pick $x_1 \in X \setminus V_{\alpha_0}$; then there exists $\alpha_1$
with $B(x_1,r_0) \subseteq V_{\alpha_1}$. We can continue to pick $x_2,x_3,\dots$ infinitely.

The $x_n$ here cannot contains any convergent subsequence, pick any $x_i, x_j$ in $\{ x_n \}$,
assume $i < j$, by $x_j \in X \setminus B(x_i, r_0)$, we got $d(x_i, x_j) \ge r_0$.
This is contradict with $X$ is sequentially compact, thus $\{V_\alpha\}$ has a finite sub cover.

    \end{enumerate}

\end{proof}


\begin{thm}\label{e0cea514}
    Suppose $I_k = [a_k, b_k]$ where $a_k \le b_k$, and $I_{k+1} \subseteq I_k$, then

    \[
        \bigcap_{k=1}^{\infty} I_k \ne \emptyset
    \]
\end{thm}

\begin{proof}
    put 
    
    \begin{align*}
        a &= \sup_{k \ge 1} a_k \\
        b &= \inf_{k \ge 1} b_k \\
    \end{align*}

    Then $\forall x \in [a,b]$, we got 

    \[
        [a,b] \subseteq \bigcap_{k=1}^{\infty}I_k
    \]
\end{proof}

\begin{corollary}
    Suppose $I_k \subseteq \mathbb{R}^n$, and

    \[
        I_k = [a^{(1)}_k, b^{(1)}_k] \times [a^{(2)}_k, b^{(2)}_k] \times \dots [a^{(n)}_k, b^{(n)}_k]
    \]

    If $I_{k+1} \subseteq I_{k}, \forall k, I_k \ne \emptyset$, then 

    \[
        \bigcap_{k=1}^{\infty} I_k \ne \emptyset
    \]
\end{corollary}

\begin{proof}
    Consider coordinates of $I_k$ and apply \cref{e0cea514}
\end{proof}

\begin{thm}[Heine-Borel]\label{heine-borel}
    In $\mathbb{R}^n$, bounded and closed set is compact. 

    We consider cube at first, assume 

    \[
        I_0 =  \{ (x_1,x_2,\dots, x_n):  0 \le x_i \le a\}
    \]

    And $I_0$ has an open cover.

    Then we divide $I_0$ into $2^n$ pieces, if $I_0$ is not compact, 
    there exists $I_1 \subseteq I_0$ which is one of piece of $I_0$, and 
    $I_1$ has no finite sub cover. And we can construct $I_1,I_2,I_3,\dots$ 
    inductively, where $I_k$ has no finite sub cover.

    Pick $p$ such that

    \[
        p \in \bigcap_{k=0}^{\infty}I_k
    \]

    Since $p \in V$ for some open set, there exists $B(p, r) \subseteq V$. 

    When $k$ is large enough, we got $I_k \subseteq B(p,r)$, because

    \[
        \mathrm{diam}(I_k) \le \sqrt{n} \frac{a}{2^{nk}}
    \]

    Which shows closed and bounded cube is compact.

    For any closed and bounded set, we can put it inside a closed and bounded cube.
    Since closed sub set of compact is closed, so closed and bounded set is closed.
\end{thm}


\begin{thm}[Bolzano-Weierstrass]\label{metric-space-bolzano-weierstrass}
    Each infinite bounded sequence in $\mathbb{R}^n$
 has a convergent subsequence. 
\end{thm}

\begin{proof}
    A bounded and closed set is a compact metric space, and hence sequentially compact.
\end{proof}

\begin{remark}
    The following three proposition is identical in $\mathbb{R}^n$, assume $K \subseteq \mathbb{R}^n$

    \begin{enumerate}
        \item $K$ is compact.
        \item $K$ is closed and bounded.
        \item $K$ is sequentially compact.
    \end{enumerate}
\end{remark}

\subsection{Exercise}

\begin{exercise}
    Let $X \ne \emptyset, f: \mathscr{P}(X) \to \mathscr{P}(X)$. If $\forall A \subseteq B, f(A) \subseteq f(B)$. Then
    exists $T \subseteq X$ and $f(T) = T$
\end{exercise}


\begin{proof}
   Define 

   \[
        T = \bigcup \{ A \subseteq X: A \subseteq f(A) \}
   \]

   Then we got $T \subseteq f(T)$ by

   \[
    f(T) = \bigcup \{ f(A): A \subseteq f(A) \} \supseteq  \bigcup \{ A: A \subseteq f(A) \}
   \]

   Since $f(T) \subseteq f(f(T))$, which shows $f(T) \subseteq T$

\end{proof}
