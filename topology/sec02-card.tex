\section{Cardinality}

\subsection{Axiom of Choice}

\begin{thm}[Axiom of Choice]\label{4ed30899}
    Let $\{A_{\alpha} : \alpha \in I\}$ be a family of nonempty sets. 
    Then there exists a function 
    \[
        f : I \to \bigcup_{\alpha \in I} A_{\alpha}
    \]
    such that $f(\alpha) \in A_{\alpha}$ for every $\alpha \in I$.
\end{thm}

\begin{thm}[Zorn's Lemma]\label{b8cf3e73}
    Let $(A,\leq)$ be a partially ordered set. 
    If every totally ordered subset of $A$ has an upper bound in $A$, 
    then $A$ contains at least one maximal element.
\end{thm}

\begin{thm}[Hausdorff Maximal Principle]\label{56941de4}
    Every partially ordered set contains a maximal totally ordered subset 
    (i.e., a maximal chain).
\end{thm}

\begin{thm}[Well-Ordering Principle]\label{7930915a}
    Every set $A$ can be equipped with a well-ordering; that is, 
    there exists a total order $\leq$ on $A$ such that every nonempty 
    subset of $A$ has a least element.
\end{thm}

\begin{thm}
\cref{b8cf3e73} (Zorn's Lemma) is equivalent to \cref{56941de4} (Hausdorff Maximal Principle).
\end{thm}

\begin{proof}
    \begin{enumerate}
        \item[($\Rightarrow$)] \textbf{Zorn $\Rightarrow$ Hausdorff.}
        
        Let $(A,\le)$ be a partially ordered set. Let $\mathscr{C}$ be the collection of all chains (totally ordered subsets) of $A$, ordered by inclusion $\subseteq$. 
        
        If $\mathscr{A}\subseteq \mathscr{C}$ is a chain with respect to $\subseteq$, then
        \[
            C^\uparrow := \bigcup_{C\in \mathscr{A}} C
        \]
        is again a chain in $A$ (any two elements of $C^\uparrow$ lie together in some $C\in\mathscr{A}$ and hence are comparable). Thus $C^\uparrow$ is an upper bound of $\mathscr{A}$ in $(\mathscr{C},\subseteq)$.
        
        By \cref{b8cf3e73} (Zorn's Lemma), $(\mathscr{C},\subseteq)$ has a maximal element $C^*$. By definition, $C^*$ is a maximal chain of $A$, which is precisely the Hausdorff maximal principle.
        
        \item[($\Leftarrow$)] \textbf{Hausdorff $\Rightarrow$ Zorn.}
        
        Suppose $(A,\le)$ is a partially ordered set in which every chain has an upper bound in $A$. By \cref{56941de4} (Hausdorff), there exists a maximal chain $C^*\subseteq A$. Let $u$ be an upper bound of $C^*$ in $A$.
        
        We claim that $u$ is a maximal element of $A$. Indeed, if $u$ were not maximal, there would exist $v\in A$ with $u<v$. Then $C^*\cup\{v\}$ is a chain (every element of $C^*$ is $\le u<v$ or $\le v$), properly containing $C^*$, which contradicts the maximality of $C^*$ as a chain. Hence $u$ is maximal, proving Zorn's Lemma.
    \end{enumerate}
\end{proof}

\subsection{cardinality}

\begin{definition}
    If $X$ and $Y$ are nonempty sets, we define the expressions
\[
\mathrm{card}(X) \leq \mathrm{card}(Y), \quad
\mathrm{card}(X) = \mathrm{card}(Y), \quad
\mathrm{card}(X) \geq \mathrm{card}(Y)
\]
to mean that there exists $f : X \to Y$ which is injective, bijective, or surjective,
respectively. We also define
\[
\mathrm{card}(X) < \mathrm{card}(Y), \quad
\mathrm{card}(X) > \mathrm{card}(Y)
\]
to mean that there is an injection but no bijection, or a surjection but no bijection,
from $X$ to $Y$. Observe that we attach no meaning to the expression
``$\mathrm{card}(X)$'' when it stands alone; there are various ways of doing so, but
they are irrelevant for our purposes (except when $X$ is finite---see below). These
relationships can be extended to the empty set by declaring that
\[
\mathrm{card}(\varnothing) < \mathrm{card}(X)
\quad\text{and}\quad
\mathrm{card}(X) > \mathrm{card}(\varnothing)
\quad\text{for all } X \neq \varnothing.
\]

For the remainder of this section we assume implicitly that all sets in question are
nonempty in order to avoid special arguments for $\varnothing$. Our first task is to prove
that the relationships defined above enjoy the properties that the notation suggests.
\end{definition}

\begin{prop}
$\mathrm{card}(X) \le \mathrm{card}(Y)$ iff $\mathrm{card}(Y) \ge \mathrm{card}(X)$
\end{prop}

\begin{proof}
($\Rightarrow$) Suppose there exists an injection $f:X \to Y$. 
Define $h:Y \to X$ by
\[
   h(y) = 
   \begin{cases}
      x, & \text{if } y = f(x) \text{ for some } x \in X, \\
      x^*, & \text{if } y \notin f(X),
   \end{cases}
\]
where $x^*\in X$ is fixed. Then for every $x\in X$, we have $h(f(x))=x$, so $h$ is surjective.


($\Leftarrow$) Suppose there exists a surjection $f:Y \to X$. 
For each $x\in X$, the fiber 
\[
   f^{-1}(\{x\}) = \{ y \in Y : f(y)=x\}
\]
is nonempty. By the Axiom of Choice, we can choose one element $h(x)\in f^{-1}(\{x\})$. 
This defines a map $h:X \to Y$, and by construction, $f(h(x))=x$ for each $x\in X$. 
Thus $h$ is injective.
\end{proof}

\begin{prop}
For any sets $X$ and $Y$, either $\mathrm{card}(X) \leq \mathrm{card}(Y)$ or 
$\mathrm{card}(Y) \leq \mathrm{card}(X)$.
\end{prop}

\begin{proof}
Consider the set $\mathcal{J}$ of all injections from subsets of $X$ to $Y$. 
The members of $\mathcal{J}$ can be regarded as subsets of $X \times Y$, so 
$\mathcal{J}$ is partially ordered by inclusion. It is easily verified that 
Zorn's lemma applies, so $\mathcal{J}$ has a maximal element $f$, with (say) 
domain $A$ and range $B$. If $x_0 \in X \setminus A$ and $y_0 \in Y \setminus B$, 
then $f$ can be extended to an injection from $A \cup \{x_0\}$ to 
$Y \cup \{y_0\}$ by setting $f(x_0) = y_0$, contradicting maximality. 
Hence either $A = X$, in which case $\mathrm{card}(X) \leq \mathrm{card}(Y)$, 
or $B = Y$, in which case $f^{-1}$ is an injection from $Y$ to $X$ and 
$\mathrm{card}(Y) \leq \mathrm{card}(X)$.
\end{proof}

\begin{thm}\label{5dd1cdb2}
    If exists $f: X \to Y$ and $g: Y \to X$, then $X$ could be divide as disjoint union $A \cup \tilde{A}$, and 
    $Y$ could be divide as disjoint union $B \cup \tilde{B}$, such that $f(A) = B,\: g(\tilde{B}) = \tilde{A}$
\end{thm}

\begin{proof}
    consider set function $h: \mathscr{P}(X) \to \mathscr{P}(X)$

    \[
        h(E) =  X \setminus g(Y \setminus f(E)) 
    \]

    Since $h$ is monotone increasing, there exists fixed point $A \subseteq X$ and

    \begin{align*}
        A &= h(A) = \bigcup \{ E: E \subseteq h(E) \} \\
    \end{align*}

    where $E \subseteq h(E)$ is equivalent to $E \cap g(Y \setminus f(E)) = \emptyset$,
    by $A = h(A)$, shows $A \cap g(Y \setminus f(A)) = \emptyset$.

    Now, define $B = f(A),\: \tilde{B} = Y \setminus B$.

    It is obviously that $A \cap g(\tilde{B}) = \emptyset$ by

    \[
        A \cap g(Y \setminus B) =  A \cap g(Y \setminus f(A)) = \emptyset
    \]

    And thus $g(\tilde{B}) \subseteq X \setminus A$, assume $x_0 \in X \setminus A$, while $x_0 \notin g(\tilde{B})$,
    Put $A_0 = A \cup \{ x_0 \}$, Then

    \begin{align*}
        A_0 \cap g(Y \setminus f(A_0)) & \subseteq A_0  \cap g(Y \setminus f(A)) \\
        & \subseteq \emptyset 
    \end{align*}

    It is contradict with $A$ is maximal set which satisfy $A \subseteq h(A)$

\end{proof}

\begin{thm}
    If $\mathrm{card}(X) \le \mathrm{card}(Y)$ and $\mathrm{card}(Y) \le \mathrm{card}(X)$, then
    $\mathrm{card}(X) = \mathrm{card}(Y)$. 
\end{thm}

\begin{proof}
    Let $f: X \to Y$ and $g: Y \to X$ both be injection.
    By \cref{5dd1cdb2}, we can divide $X$ as disjoint union $A \cup \tilde{A}$, and $Y$
    as disjoin union $B \cup \tilde{B}$, such that $f(A) = B,\: g(\tilde{B}) = \tilde{A}$.

    Now, let's define $h: X \to Y$ as

    \begin{align*}
        h(x) = \begin{cases}
            f(x) & x \in A\\
            g^{-1}(x) & x \in \tilde{A}
        \end{cases}
    \end{align*}

    This is a well define because $g$ is bijection on $\tilde{B} \to \tilde{A}$, hence has inverse.

    Then $h$ is injection, since $f(A) \cap g^{-1}(\tilde{A}) = B \cap \tilde{B} = \emptyset$. $h$
    is surjection by $h(X) = h(A \cup \tilde{A}) = h(A) \cup h(\tilde{A}) = B \cup \tilde{B} = Y$
\end{proof}

\begin{prop}
For any set $X$, $\mathrm{card}(X) < \mathrm{card}(\mathscr{P}(X))$
\end{prop}

\begin{proof}
    ($\le$) Define $h: X \to \mathscr{P}(X)$ where $h(x) = \{x\}$, then $h$ is injection obviously.


(strictness) Suppose, for contradiction, there is a bijection $f:X\to\mathscr{P}(X)$
(hence a surjection). Define the diagonal set
\[
T=\{x\in X: x\notin f(x)\}.
\]
Since $f$ is surjective, there exists $x^*\in X$ with $f(x^*)=T$.

Now:

\begin{enumerate}
    \item If $x^*\in T$, then by definition of $T$ we must have $x^*\notin f(x^*)=T$,
  a contradiction.

    \item If $x^*\notin T$, then by definition of $T$ we must have $x^*\in f(x^*)=T$,
  a contradiction.

\end{enumerate}
Both cases are impossible. Therefore no bijection $X\to\mathscr{P}(X)$ exists, so
$\mathrm{card}(X)<\mathrm{card}(\mathscr{P}(X))$.
\end{proof}


\begin{thm}\label{8e545ca3}
    If set $A$ is infinite, then $\mathrm{card}(\mathbb{N}) \le \mathrm{card}(A)$
\end{thm}

\begin{proof}
    We can pick $a_0,a_1,a_2,\dots$ from $A$, and define surjection $f: A \to \mathbb{N}$ as

    \[
        f(a_i) = i
    \]
\end{proof}

\begin{thm}
    If $A_1,A_2,\dots, A_n$ is a finite sequence of countable sets. Then

    \[
        A = A_1 \times A_2 \times \dots \times A_n
    \]

    is countable
\end{thm}

\begin{proof}
    $A$ is infinite and $\mathrm{card}(\mathbb{N}) \le \mathrm{card}(A)$ by \cref{8e545ca3}.
\end{proof}

\subsection{Exercise}

\begin{exercise}
    Let $f: \mathbb{R} \to \mathbb{R}$ be continuous and injective, then $f$ is strictly monotone.
\end{exercise}

\begin{proof}
    We will prove at first, if $x_0 < y_0$, $f(x_0) < f(y_0)$, then $\forall x \in (x_0, y_0),\: f(x_0) < f(x) < f(y)$.

    For contradiction, assume exists $x \in (x_0, y_0),\: f(x) < f(x_0)$.

    Put

    \begin{align*}
        c &= \frac{f(x) + f(x_0)}{2} \\
        c & \in [f(x), f(x_0)] \cap [f(x), f(y_0)] 
    \end{align*}

    by intermediate theorem, there exists $y \in [x_0, x],\: f(y) = c$.
    And $y_1 \in [x,y_0], f(y_1) = c$.

    By $y < y_1$ (Otherwise $f(x) = c$, impossible), we got $f$ no longer be injective, shows 
    contradict with $f(x) < f(x_0)$.

    Similarly for $f(x) > f(y_0)$.

    Now we can prove $f$ is strictly increasing on any $[x_0, y_0]$ if $f(x_0) < f(y_0)$. Pick $x <y,\: x,y \in (x_0, y_0)$. By $y \in (x_0, y_0)$,
    we got $f(x_0) < f(y) < f(y_0)$, by $x \in (x_0, y)$, we got $f(x_0) < f(x) < f(y)$.

    Further more, we can extend $[x_0, y_0]$ to $[x_0, y_0 + \delta]$ by any $\delta > 0$,
    and keep $f(x_0) < f(y_0 + \delta)$. Because if $f(y_0 + \delta) < f(x_0)$, 
    shows $-f(x_0) < - f(y_0 + \delta)$, but $-f(y_0) < -f(x_0)$, be contradict with our previous conclusion. 

    Similarly, we can extend $[x_0, y_0]$ to $[x_0 - \delta, y_0]$. Combine above, we can extend $[x_0, y_0]$
    to cover every point in $\mathbb{R}$.


\end{proof}

\begin{exercise}
Let $f:\mathbb{R}\to\mathbb{R}$ be continuous. If $f$ is injective on the set
of irrationals $\mathbb{R}\setminus\mathbb{Q}$, then $f$ is injective on $\mathbb{Q}$ as well.
Equivalently, there is no continuous $f$ that is one-to-one on $\mathbb{R}\setminus\mathbb{Q}$
but not one-to-one on $\mathbb{Q}$.
\end{exercise}

\begin{proof}
Assume for contradiction that there exist distinct rationals $q_1<q_2$ with
$f(q_1)=f(q_2)=:v$. Because $f$ is continuous on the compact interval $[q_1,q_2]$,
it attains a maximum $M$ and a minimum $m$ there.

If $f$ were constant on $[q_1,q_2]$, then $f(x)=v$ for all $x\in[q_1,q_2]$.
In particular, choosing two distinct irrationals $x\neq y$ in $(q_1,q_2)$ would give
$f(x)=f(y)$ with $x,y\in\mathbb{R}\setminus\mathbb{Q}$, contradicting injectivity on
the irrationals. Hence $f$ is not constant on $[q_1,q_2]$, so either $M>v$ or $m<v$.
Without loss of generality, suppose $M>v$, and let $c\in[q_1,q_2]$ satisfy $f(c)=M$.

Consider the open interval $(v,M)$. Since $f(\mathbb{Q})$ is at most countable while $(v,M)$
is uncountable, we may choose a real number $y\in(v,M)\setminus f(\mathbb{Q})$.
By the Intermediate Value Theorem applied to $[q_1,c]$ and to $[c,q_2]$, there exist points
$x_1\in(q_1,c)$ and $x_2\in(c,q_2)$ with
\[
f(x_1)=y=f(x_2).
\]
Because $y\notin f(\mathbb{Q})$, neither $x_1$ nor $x_2$ can be rational.
Thus $x_1,x_2\in\mathbb{R}\setminus\mathbb{Q}$, and $x_1\neq x_2$.
This contradicts the injectivity of $f$ on $\mathbb{R}\setminus\mathbb{Q}$.

Therefore our assumption was false, and $f$ must be injective on $\mathbb{Q}$.
\end{proof}

\begin{exercise}
    Let $f: X \to Y,\: g: Y \to X$, if $\forall x \in X, g(f(x)) = x$, then $g$ is surjective 
    while $f$ is injective.
\end{exercise}

\begin{proof}
   $f$ is injective obviously, otherwise if $f(x_1) = f(x_2) = y$, then $h(x_1) = h(x_2) = g(y)$, which is contradict with
   $h$ is injective.

   $g$ is also surjective, if exists $x \in X$ and $x \notin g(Y)$. Then we got $h(x) \ne x$ since $x \notin g(Y)$
\end{proof}

\begin{exercise}
    Any vector space has a basis.
\end{exercise}

\begin{proof}
   Consider vector space $V$, which is not zero space. Let $E$ contains  
   all linearly independent subset of $V$. Consider $\subseteq$ as partial order.
   For any total order chain $C$, define 

   \[
        W = \bigcup_{U \in C} U
   \]

   We will show $W$ is linearly independent. Assume $x_1, x_2, \dots, x_n \in W$, and exists $c_1,c_2,\dots c_n$ such that

   \[
    c_1x_1 + c_2 x_2 + \dots + c_n x_n = 0
   \]

    Since $x_i \in U_i,\: 1 \le i \le n$, for some linearly independent subset $U_i$. 
    By $W$ is totally ordered, there exists $U'$ which is the 
    maximum element in $U_1,U_2,\dots ,U_n$. Since $x_i \in U', \: \forall 1 \le i \le n$, 
    and $U'$ is linearly independent, we got $c_1 = c_2 = \dots = c_n = 0$

\end{proof}