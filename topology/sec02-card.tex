\section{Cardinality}

\subsection{Axiom of Choice}

\begin{thm}[Axiom of Choice]\label{4ed30899}
    Let $\{A_{\alpha} : \alpha \in I\}$ be a family of nonempty sets. 
    Then there exists a function 
    \[
        f : I \to \bigcup_{\alpha \in I} A_{\alpha}
    \]
    such that $f(\alpha) \in A_{\alpha}$ for every $\alpha \in I$.
\end{thm}

\begin{thm}[Zorn's Lemma]\label{b8cf3e73}
    Let $(A,\leq)$ be a partially ordered set. 
    If every totally ordered subset of $A$ has an upper bound in $A$, 
    then $A$ contains at least one maximal element.
\end{thm}

\begin{thm}[Hausdorff Maximal Principle]\label{56941de4}
    Every partially ordered set contains a maximal totally ordered subset 
    (i.e., a maximal chain).
\end{thm}

\begin{thm}[Well-Ordering Principle]\label{7930915a}
    Every set $A$ can be equipped with a well-ordering; that is, 
    there exists a total order $\leq$ on $A$ such that every nonempty 
    subset of $A$ has a least element.
\end{thm}

\begin{thm}
\cref{b8cf3e73} (Zorn's Lemma) is equivalent to \cref{56941de4} (Hausdorff Maximal Principle).
\end{thm}

\begin{proof}
    \begin{enumerate}
        \item[($\Rightarrow$)] \textbf{Zorn $\Rightarrow$ Hausdorff.}
        
        Let $(A,\le)$ be a partially ordered set. Let $\mathcal{C}$ be the collection of all chains (totally ordered subsets) of $A$, ordered by inclusion $\subseteq$. 
        
        If $\mathcal{A}\subseteq \mathcal{C}$ is a chain with respect to $\subseteq$, then
        \[
            C^\uparrow := \bigcup_{C\in \mathcal{A}} C
        \]
        is again a chain in $A$ (any two elements of $C^\uparrow$ lie together in some $C\in\mathcal{A}$ and hence are comparable). Thus $C^\uparrow$ is an upper bound of $\mathcal{A}$ in $(\mathcal{C},\subseteq)$.
        
        By \cref{b8cf3e73} (Zorn's Lemma), $(\mathcal{C},\subseteq)$ has a maximal element $C^*$. By definition, $C^*$ is a maximal chain of $A$, which is precisely the Hausdorff maximal principle.
        
        \item[($\Leftarrow$)] \textbf{Hausdorff $\Rightarrow$ Zorn.}
        
        Suppose $(A,\le)$ is a partially ordered set in which every chain has an upper bound in $A$. By \cref{56941de4} (Hausdorff), there exists a maximal chain $C^*\subseteq A$. Let $u$ be an upper bound of $C^*$ in $A$.
        
        We claim that $u$ is a maximal element of $A$. Indeed, if $u$ were not maximal, there would exist $v\in A$ with $u<v$. Then $C^*\cup\{v\}$ is a chain (every element of $C^*$ is $\le u<v$ or $\le v$), properly containing $C^*$, which contradicts the maximality of $C^*$ as a chain. Hence $u$ is maximal, proving Zorn's Lemma.
    \end{enumerate}
\end{proof}

\begin{thm}
    \cref{b8cf3e73} (Zorn's Lemma) implies \cref{7930915a} (Well-Ordering Principle)
\end{thm}

\begin{proof}
    Given any set $A$,

    Consider

    \[
        \mathcal{W} = \{(B, \le): B \subseteq A,\: \le \text{ a well order on} B \}
    \]

    For any well-ordered subsets $(B_1, \le_1)$ and $(B_2, \le_2)$, we define a partial order $\subseteq_*$
    between them as follows: $(B_1, \le_1) \subseteq_* (B_2, \le_2)$ iff:

    \begin{enumerate}
        \item $B_1 \subseteq B_2$
        \item $\forall x,y \in B_1,\: x \le_1 y $ implies $x \le_2 y$
        \item $\forall x \in B_1, y \in B_2 \setminus B_1,\: x \le_2 y$
    \end{enumerate}

    For any chain $C = \{ (B_i, \le_i) \}_{i \in I}$ under $\subseteq_*$, we can obtain an upper bound $(B^*, \le_*)$ by defining:

    \begin{align*}
        B^* &= \bigcup_{i \in I} B_i \\
        x \le_* y &\iff \exists i \in I \text{ such that } x,y \in B_i \text{ and } x \le_i y
    \end{align*}

    Then $(B^*, \le_*)$ is well-ordered: 

    \begin{enumerate}
        \item \textbf{Totally ordered:} 
        
        For any $x,y \in B^*$, there exist indices $i, j \in I$ such that $x \in B_i$ and $y \in B_j$. 
        Since $C$ is a chain under $\subseteq_*$, either $(B_i, \le_i) \subseteq_* (B_j, \le_j)$ or 
        $(B_j, \le_j) \subseteq_* (B_i, \le_i)$. Without loss of generality, assume the former. 
        Then $x \in B_i \subseteq B_j$, so both $x, y \in B_j$. Since $\le_j$ is a total order on $B_j$, 
        either $x \le_j y$ or $y \le_j x$, which means either $x \le_* y$ or $y \le_* x$.

        \item \textbf{Every nonempty subset has a minimum element:} 
        
        Let $E \subseteq B^*$ be nonempty. Pick any $a \in E$; then there exists $i \in I$ such that $a \in B_i$, 
        and hence $E \cap B_i \neq \emptyset$. Since $(B_i, \le_i)$ is well-ordered, the nonempty set $E \cap B_i$ 
        has a minimum element $a_0$ under $\le_i$. 

        We claim that $a_0$ is the minimum of $E$ under $\le_*$. For any $a' \in E$:
        \begin{itemize}
            \item If $a' \in B_i$, then $a_0 \le_i a'$ since $a_0$ is the minimum of $E \cap B_i$, so $a_0 \le_* a'$.
            \item If $a' \notin B_i$, then $a' \in B_j$ for some $j \in I$. Since $C$ is a chain, either 
            $(B_i, \le_i) \subseteq_* (B_j, \le_j)$ or $(B_j, \le_j) \subseteq_* (B_i, \le_i)$. 
            In the first case, $a_0 \in B_i \subseteq B_j$ and by condition (3) of $\subseteq_*$, we have $a_0 \le_j a'$, 
            so $a_0 \le_* a'$. In the second case, $a' \in B_j \subseteq B_i$, contradicting $a' \notin B_i$.
        \end{itemize}
        
        Therefore, $a_0 \le_* a'$ for all $a' \in E$, making $a_0$ the minimum element of $E$.
    \end{enumerate}

    Thus, $(B^*, \le_*)$ is an upper bound of the chain $C$. By \cref{b8cf3e73} (Zorn's Lemma), 
    there exists a maximal element $(B^{s}, \le_{s})$ under $\subseteq_*$.

    We claim that $B^s = A$. Suppose for contradiction that there exists $a \in A \setminus B^s$. 
    We can then extend $(B^s, \le_s)$ to $(B^s \cup \{ a \}, \le_{t})$ by defining $\le_{t}$ as follows:
    \begin{align*}
        x \le_t y &\iff x \le_s y \quad \text{for all } x,y \in B^s \\
        x \le_t a &\text{ for all } x \in B^s \\
        a \le_t y &\iff \text{false for all } y \in B^s
    \end{align*}
    
    This makes $a$ the maximum element of $B^s \cup \{a\}$ under $\le_t$. It is straightforward to verify that 
    $\le_t$ is a well-ordering on $B^s \cup \{a\}$ and that $(B^s, \le_s) \subseteq_* (B^s \cup \{a\}, \le_t)$. 
    This contradicts the maximality of $(B^s, \le_s)$.
    
    Therefore, $B^s = A$, and $(A, \le_s)$ is well-ordered.

\end{proof}

\begin{thm}
    \cref{7930915a} (Well-Ordering Principle) implies \cref{4ed30899} (Axiom of Choice).
\end{thm}

\begin{proof}
    Given a family of nonempty sets $\{A_{\alpha}: \alpha \in I \}$. Since $A_{\alpha}$ could be well-ordered by 
    relation $\le_{\alpha}$, thus we define $f: \alpha \to \bigcup_{\alpha \in I} A_{\alpha}$ as

    \[
        f(\alpha) = \min \{ A_{\alpha} \} \quad \text{by } \le_{\alpha} 
    \]
\end{proof}

\begin{thm}
    \cref{4ed30899} (Axiom of Choice) implies \cref{b8cf3e73} (Zorn's Lemma).
\end{thm}

\begin{proof}
    For the sake of contradiction, assume $A$ does not contain any maximal element.
    By Axiom of Choice, there exists function $f: A \to A$ and $\forall a \in A,\: a < f(a)$.

    We will construct a injection from Ord $\to A$, and this will violates that Ord is a proper class.

    To construct such injection, we will start with $\alpha_0 \in \mathrm{Ord}$ by picking any $a_0 \in A$.

    We define the following, where $\lambda$ denotes any limit ordinal:

    \begin{align*}
        h(\alpha_0) &= a_0 \\
        h(\alpha + 1) &= f(h(\alpha)) \\
        h(\lambda) & = f(u) \quad u \text{ is upper bound of } \{ h(\beta): \beta < \lambda \}
    \end{align*}

    $h$ is well defined, since $\{ h(\beta): \beta < \lambda \}$ is totally ordered, and hence has an upper bound.

    After all, the injection $h$ violates that Ord is a proper class, so $A$ must have a maximal element.
\end{proof}

\subsection{cardinality}

\begin{definition}
    If $X$ and $Y$ are nonempty sets, we define the expressions
\[
\mathrm{card}(X) \leq \mathrm{card}(Y), \quad
\mathrm{card}(X) = \mathrm{card}(Y), \quad
\mathrm{card}(X) \geq \mathrm{card}(Y)
\]
to mean that there exists $f : X \to Y$ which is injective, bijective, or surjective,
respectively. We also define
\[
\mathrm{card}(X) < \mathrm{card}(Y), \quad
\mathrm{card}(X) > \mathrm{card}(Y)
\]
to mean that there is an injection but no bijection, or a surjection but no bijection,
from $X$ to $Y$. Observe that we attach no meaning to the expression
``$\mathrm{card}(X)$'' when it stands alone; there are various ways of doing so, but
they are irrelevant for our purposes (except when $X$ is finite---see below). These
relationships can be extended to the empty set by declaring that
\[
\mathrm{card}(\varnothing) < \mathrm{card}(X)
\quad\text{and}\quad
\mathrm{card}(X) > \mathrm{card}(\varnothing)
\quad\text{for all } X \neq \varnothing.
\]

For the remainder of this section we assume implicitly that all sets in question are
nonempty in order to avoid special arguments for $\varnothing$. Our first task is to prove
that the relationships defined above enjoy the properties that the notation suggests.
\end{definition}

\begin{prop}
$\mathrm{card}(X) \le \mathrm{card}(Y)$ iff $\mathrm{card}(Y) \ge \mathrm{card}(X)$
\end{prop}

\begin{proof}
($\Rightarrow$) Suppose there exists an injection $f:X \to Y$. 
Define $h:Y \to X$ by
\[
   h(y) = 
   \begin{cases}
      x, & \text{if } y = f(x) \text{ for some } x \in X, \\
      x^*, & \text{if } y \notin f(X),
   \end{cases}
\]
where $x^*\in X$ is fixed. Then for every $x\in X$, we have $h(f(x))=x$, so $h$ is surjective.


($\Leftarrow$) Suppose there exists a surjection $f:Y \to X$. 
For each $x\in X$, the fiber 
\[
   f^{-1}(\{x\}) = \{ y \in Y : f(y)=x\}
\]
is nonempty. By the Axiom of Choice, we can choose one element $h(x)\in f^{-1}(\{x\})$. 
This defines a map $h:X \to Y$, and by construction, $f(h(x))=x$ for each $x\in X$. 
Thus $h$ is injective.
\end{proof}

\begin{prop}
For any sets $X$ and $Y$, either $\mathrm{card}(X) \leq \mathrm{card}(Y)$ or 
$\mathrm{card}(Y) \leq \mathrm{card}(X)$.
\end{prop}

\begin{proof}
Consider the set $\mathcal{J}$ of all injections from subsets of $X$ to $Y$. 
The members of $\mathcal{J}$ can be regarded as subsets of $X \times Y$, so 
$\mathcal{J}$ is partially ordered by inclusion. It is easily verified that 
Zorn's lemma applies, so $\mathcal{J}$ has a maximal element $f$, with (say) 
domain $A$ and range $B$. If $x_0 \in X \setminus A$ and $y_0 \in Y \setminus B$, 
then $f$ can be extended to an injection from $A \cup \{x_0\}$ to 
$Y \cup \{y_0\}$ by setting $f(x_0) = y_0$, contradicting maximality. 
Hence either $A = X$, in which case $\mathrm{card}(X) \leq \mathrm{card}(Y)$, 
or $B = Y$, in which case $f^{-1}$ is an injection from $Y$ to $X$ and 
$\mathrm{card}(Y) \leq \mathrm{card}(X)$.
\end{proof}

\begin{thm}\label{5dd1cdb2}
    If exists $f: X \to Y$ and $g: Y \to X$, then $X$ could be divide as disjoint union $A \cup \tilde{A}$, and 
    $Y$ could be divide as disjoint union $B \cup \tilde{B}$, such that $f(A) = B,\: g(\tilde{B}) = \tilde{A}$
\end{thm}

\begin{proof}
    consider set function $h: \mathscr{P}(X) \to \mathscr{P}(X)$

    \[
        h(E) =  X \setminus g(Y \setminus f(E)) 
    \]

    Since $h$ is monotone increasing, there exists fixed point $A \subseteq X$ and

    \begin{align*}
        A &= h(A) = \bigcup \{ E: E \subseteq h(E) \} \\
    \end{align*}

    where $E \subseteq h(E)$ is equivalent to $E \cap g(Y \setminus f(E)) = \emptyset$,
    by $A = h(A)$, shows $A \cap g(Y \setminus f(A)) = \emptyset$.

    Now, define $B = f(A),\: \tilde{B} = Y \setminus B$.

    It is obviously that $A \cap g(\tilde{B}) = \emptyset$ by

    \[
        A \cap g(Y \setminus B) =  A \cap g(Y \setminus f(A)) = \emptyset
    \]

    And thus $g(\tilde{B}) \subseteq X \setminus A$, assume $x_0 \in X \setminus A$, while $x_0 \notin g(\tilde{B})$,
    Put $A_0 = A \cup \{ x_0 \}$, Then

    \begin{align*}
        A_0 \cap g(Y \setminus f(A_0)) & \subseteq A_0  \cap g(Y \setminus f(A)) \\
        & \subseteq \emptyset 
    \end{align*}

    It is contradict with $A$ is maximal set which satisfy $A \subseteq h(A)$

\end{proof}

\begin{thm}
    If $\mathrm{card}(X) \le \mathrm{card}(Y)$ and $\mathrm{card}(Y) \le \mathrm{card}(X)$, then
    $\mathrm{card}(X) = \mathrm{card}(Y)$. 
\end{thm}

\begin{proof}
    Let $f: X \to Y$ and $g: Y \to X$ both be injection.
    By \cref{5dd1cdb2}, we can divide $X$ as disjoint union $A \cup \tilde{A}$, and $Y$
    as disjoin union $B \cup \tilde{B}$, such that $f(A) = B,\: g(\tilde{B}) = \tilde{A}$.

    Now, let's define $h: X \to Y$ as

    \begin{align*}
        h(x) = \begin{cases}
            f(x) & x \in A\\
            g^{-1}(x) & x \in \tilde{A}
        \end{cases}
    \end{align*}

    This is a well define because $g$ is bijection on $\tilde{B} \to \tilde{A}$, hence has inverse.

    Then $h$ is injection, since $f(A) \cap g^{-1}(\tilde{A}) = B \cap \tilde{B} = \emptyset$. $h$
    is surjection by $h(X) = h(A \cup \tilde{A}) = h(A) \cup h(\tilde{A}) = B \cup \tilde{B} = Y$
\end{proof}

\begin{prop}
For any set $X$, $\mathrm{card}(X) < \mathrm{card}(\mathscr{P}(X))$
\end{prop}

\begin{proof}
    ($\le$) Define $h: X \to \mathscr{P}(X)$ where $h(x) = \{x\}$, then $h$ is injection obviously.


(strictness) Suppose, for contradiction, there is a bijection $f:X\to\mathscr{P}(X)$
(hence a surjection). Define the diagonal set
\[
T=\{x\in X: x\notin f(x)\}.
\]
Since $f$ is surjective, there exists $x^*\in X$ with $f(x^*)=T$.

Now:

\begin{enumerate}
    \item If $x^*\in T$, then by definition of $T$ we must have $x^*\notin f(x^*)=T$,
  a contradiction.

    \item If $x^*\notin T$, then by definition of $T$ we must have $x^*\in f(x^*)=T$,
  a contradiction.

\end{enumerate}
Both cases are impossible. Therefore no bijection $X\to\mathscr{P}(X)$ exists, so
$\mathrm{card}(X)<\mathrm{card}(\mathscr{P}(X))$.
\end{proof}


\begin{thm}\label{8e545ca3}
    If set $A$ is infinite, then $\mathrm{card}(\mathbb{N}) \le \mathrm{card}(A)$
\end{thm}

\begin{proof}
    We can pick $a_0,a_1,a_2,\dots$ from $A$, and define surjection $f: A \to \mathbb{N}$ as

    \[
        f(a_i) = i
    \]
\end{proof}

\begin{thm}
    If $A_1,A_2,\dots, A_n$ is a finite sequence of countable sets. Then

    \[
        A = A_1 \times A_2 \times \dots \times A_n
    \]

    is countable
\end{thm}

\begin{proof}
    We just need to prove $A_1 \times A_2$ is countable, then finite case could be proved by induction.
    $A_1 \times A_2$ is infinite and $\mathrm{card}(\mathbb{N}) \le \mathrm{card}(A_1 \times A_2)$ by \cref{8e545ca3}.
    And we can make a injection from $A_1 \times A_2 \to \mathbb{N}$ by

    \[
        f(a_n, b_m) = \binom{n+m + 1}{2} + m
    \]

    Consider when $n_1 + m_1 > n_2 + m_2$

    \begin{align*}
        \binom{n_1 + m_1 + 1}{2} -\binom{n_2 + m_2 + 1}{2} + m_1 - m_2 & \ge \binom{n_1 + m_1+1}{2} -\binom{n_1 + m_1}{2} + m_1 - m_2 \\
        & \ge n_1 + m_1  + m_1 - m_2 \ge n_2 + m_2  + 1 + m_1 - m_2 \\
        & \ge n_2 + m_1 +1 > 0
    \end{align*}

    when $n_1 + m_1 = n_2 + m_2$, must have $m_1 \ne m_2$

    After all $f$ is injection, and hence $\mathrm{card}(A_1 \times A_2) \le \mathbb{N}$
\end{proof}

\begin{thm}
    Countable union of countable sets is countable.
\end{thm}

\begin{proof}
    Assume $A_0,A_1,\dots $ is a sequence of countable sets. Define

    \[
        A = \bigcup_{i=0}^{\infty} A_i
    \]

    And there exists a bijection $f_i: \mathbb{N} \to A_i$ for every $A_i$

    and $h: \mathbb{N} \times \mathbb{N} \to A$ as:

    \[
        h(n,m) = f_n (m)
    \]

    Then $h$ is a surjection, by $\mathrm{card}(\mathbb{N} \times \mathbb{N}) = \mathrm{card}(\mathbb{N})$ , and
    $\mathrm{card}(A) \le \mathrm{card}(\mathbb{N} \times \mathbb{N})$, we got
    $\mathrm{card}(A) = \mathrm{card}(\mathbb{N})$
\end{proof}

\begin{definition}[cardinality of the continuum]
    The cardinality of the continuum $\mathfrak{c}$ is defined as $\mathrm{card}(\mathbb{R})$
\end{definition}

\begin{thm}
    The cardinality of $\mathbb{N}$'s power set $\mathcal{P}$ is $\mathfrak{c}$
\end{thm}

\begin{proof}
    We will first show $\mathrm{card}(\mathcal{P}) = \mathrm{card}([0,1])$, since $\mathrm{card}([0,1]) = \mathrm{card}(\mathbb{R})$ by function

    \[
        h(x) = \frac{\mathrm{tanh}(x) + 1}{2}
    \]

    which is a bijection between $\mathbb{R}$ and $(0,1)$, then $\mathrm{card}([0,1]) \le \mathrm{card}(\mathbb{R})$
    by  injective identity function, and $\mathrm{card}(\mathbb{R}) = \mathrm{card}((0,1)) \le \mathrm{card}([0,1])$ 
    by $h$ and identity.

    For any $E \subseteq \mathbb{N}$, define $g: \mathcal{P} \to \mathbb{R}$ as

    \[
        g(E) = \sum_{n \in E} 10^{-n}
    \]

    Then $g$ is a injection, for any $E_1 \ne E_2$ we can pick $n = \min ( E_1 \Delta E_2 )$, and 
    
    \[
        \left| g(E_1) - g(E_2)\right| \ge 10^{-n} - 10^{-n-1} \cdot \frac{10}{9} \ge 10^{-n} \cdot \frac{8}{9} > 0
    \]

    which shows $\mathrm{card}(\mathcal{P}) \le \mathrm{card}(\mathbb{R})$

    We can also define another injection $g'(x): [0,1] \to \mathcal{P}$ as

    \begin{align*}
        a_0 &= [x],\: r_0 = 2(x - a_0) \\
        a_1 &= [r_0],\: r_1 = 2(r_0 - a_1) \\
        \dots & \\
        g'(x) &= \{ i: a_i = 1 \} \\
        x &= a_0  + a_1 c^{1} + a_2 c^{2} + \dots  \quad c = \frac{1}{2}
    \end{align*}

     The mapping $g'(x)$ is injective, i.e., for $x \neq y$ in $[0,1]$, we have $g'(x) \neq g'(y)$. However, in general, $g'(x) \subseteq g'(y)$ does not hold for $x < y$.

\end{proof}

\begin{thm}
    Assume $\mathcal{P}$ is power set of $\mathbb{N}$

    \begin{enumerate}
        \item $\mathrm{card}(\mathcal{P} \times \mathcal{P}) = \mathrm{card}(\mathcal{P})$

        \item If $\mathrm{card}(A) \le \mathfrak{c}$ and $\forall \alpha \in A, \mathrm{card}(X_{\alpha}) \le \mathfrak{c}$ Then 
        
        \[
        \mathrm{card} \Bigg( \bigcup_{\alpha \in A} X_{\alpha} \Bigg) \le \mathfrak{c} 
        \]
        
    \end{enumerate}
\end{thm}

\begin{proof}
    \begin{enumerate}
        \item  Define $\varphi: \mathbb{N} \to \mathbb{N}, \varphi(n) = 2n$ and $\phi: \mathbb{N} \to \mathbb{N}, \phi(n) = 2n + 1$
    and $f: \mathcal{P} \times \mathcal{P} \to \mathcal{P}$, where $f(A, B) = \varphi(A) \cup \phi(B)$

        \item Define

        \[
            X = \bigcup_{\alpha \in A}X_{\alpha}
        \]

        For every $x \in X$ exists $\alpha \in I$ such that $x \in X_{\alpha}$, by \cref{4ed30899} (axiom of choice).
        There exists $\beta: X \to A $ and $x \in X_{\beta(x)}$.

        By $\mathrm{card}(X_{\alpha}) \le \mathfrak{c}$ and $\mathrm{card}(A) \le \mathbb{R}$, there exists injection $g_{\alpha}: X_{\alpha} \to \mathbb{R}$
        and injection $f: A \to \mathbb{R}$.

        And we define $h: X \to \mathbb{R} \times \mathbb{R}$ as

        \[
            h(x) = \Big( f(\beta(x)), g_{\beta(x)}(x) \Big)
        \]

        Assume $x_1 \ne x_2$, one of $\beta(x_1) \ne \beta(x_2),\: g_{\beta(x_1)}(x_1) \ne g_{\beta{(x_2)}}(x_2)$ should satisfy.
        So  $h$ is a injection from $X$ to $\mathbb{R} \times \mathbb{R}$, and hence

        \[
            \mathrm{card}(X) \le \mathfrak{c}
        \]
    \end{enumerate}
\end{proof}

\begin{corollary}
    $\mathrm{card}(\mathbb{R}^2) = \mathfrak{c}$
\end{corollary}

\subsection{Well Ordered Sets}

\begin{thm}[The Principle of Transfinite Induction]
    Let $X$ be a well ordered set. If $A$ is a subset
    of $X$ such that $x \in A$ whenever $I_x \subseteq A$, then $A = X$

    $I_x$ here is defined as:

    \[
        I_x = \{ y \in X: y < x  \}
    \]
\end{thm}

\begin{proof}
    Assume $A \ne X$, and thus $X \setminus A \ne \emptyset$, pick 

    \[
        x_0 = \min X \setminus A
    \]

    By $I_{x_0} \subseteq A$, we got $x_0 \in A$, which contradicts $x_0 \in X \setminus A$.
\end{proof}

\begin{example}
    Consider $X = \{ 0,1,2, \dots \}$ and $A = \{ 1,2,3, \dots \}$. 

    The $A$ does not meet $\forall x \in X, I_x \subseteq A \Rightarrow x \in A$. Because $I_0$
    is empty set when $x=0$, and $I_0 \subseteq A$ while $0 \notin A$.
\end{example}

\begin{thm}
    If $X$ is well ordered and $A \subseteq X$, then $\bigcup_{x \in A} I_x$
    is either an initial segment or $X$ itself.
\end{thm}

\begin{proof}
    Let 
    
    \begin{align*}
        J &= \bigcup_{x \in A} I_x \\
    \end{align*}

    If $J \ne X$, let $b = \min X \setminus J$, $I_b = \{ y \in X:  y <  b\}$. 
    By definition of $b$, we got $X \setminus J \subseteq \{ y \in X: y \ge b\}$
    and hence $I_b \subseteq J$


        \begin{align*}
            J \setminus I_b &= \bigcup_{x \in A} I_x \cap \{ y \in X: y \ge b\} \\
            &= \bigcup_{x \in A} \{y \in X: b \le y < x\}
        \end{align*}

    If exists $y \in J \setminus I_b$, there exists $x \in A$ and $b \le y < x$, which shows $b \in I_x$, which 
    contradicts $b \notin J$, so $J \setminus I_b = \emptyset$.

\end{proof}

\subsection{Exercise}

\begin{exercise}
    Let $f: \mathbb{R} \to \mathbb{R}$ be continuous and injective, then $f$ is strictly monotone.
\end{exercise}

\begin{proof}
    We will prove at first, if $x_0 < y_0$, $f(x_0) < f(y_0)$, then $\forall x \in (x_0, y_0),\: f(x_0) < f(x) < f(y)$.

    For contradiction, assume exists $x \in (x_0, y_0),\: f(x) < f(x_0)$.

    Put

    \begin{align*}
        c &= \frac{f(x) + f(x_0)}{2} \\
        c & \in [f(x), f(x_0)] \cap [f(x), f(y_0)] 
    \end{align*}

    by intermediate theorem, there exists $y \in [x_0, x],\: f(y) = c$.
    And $y_1 \in [x,y_0], f(y_1) = c$.

    By $y < y_1$ (Otherwise $f(x) = c$, impossible), we got $f$ no longer be injective, shows 
    contradict with $f(x) < f(x_0)$.

    Similarly for $f(x) > f(y_0)$.

    Now we can prove $f$ is strictly increasing on any $[x_0, y_0]$ if $f(x_0) < f(y_0)$. Pick $x <y,\: x,y \in (x_0, y_0)$. By $y \in (x_0, y_0)$,
    we got $f(x_0) < f(y) < f(y_0)$, by $x \in (x_0, y)$, we got $f(x_0) < f(x) < f(y)$.

    Further more, we can extend $[x_0, y_0]$ to $[x_0, y_0 + \delta]$ by any $\delta > 0$,
    and keep $f(x_0) < f(y_0 + \delta)$. Because if $f(y_0 + \delta) < f(x_0)$, 
    shows $-f(x_0) < - f(y_0 + \delta)$, but $-f(y_0) < -f(x_0)$, be contradict with our previous conclusion. 

    Similarly, we can extend $[x_0, y_0]$ to $[x_0 - \delta, y_0]$. Combine above, we can extend $[x_0, y_0]$
    to cover every point in $\mathbb{R}$.


\end{proof}

\begin{exercise}
Let $f:\mathbb{R}\to\mathbb{R}$ be continuous. If $f$ is injective on the set
of irrationals $\mathbb{R}\setminus\mathbb{Q}$, then $f$ is injective on $\mathbb{Q}$ as well.
Equivalently, there is no continuous $f$ that is one-to-one on $\mathbb{R}\setminus\mathbb{Q}$
but not one-to-one on $\mathbb{Q}$.
\end{exercise}

\begin{proof}
Assume for contradiction that there exist distinct rationals $q_1<q_2$ with
$f(q_1)=f(q_2)=:v$. Because $f$ is continuous on the compact interval $[q_1,q_2]$,
it attains a maximum $M$ and a minimum $m$ there.

If $f$ were constant on $[q_1,q_2]$, then $f(x)=v$ for all $x\in[q_1,q_2]$.
In particular, choosing two distinct irrationals $x\neq y$ in $(q_1,q_2)$ would give
$f(x)=f(y)$ with $x,y\in\mathbb{R}\setminus\mathbb{Q}$, contradicting injectivity on
the irrationals. Hence $f$ is not constant on $[q_1,q_2]$, so either $M>v$ or $m<v$.
Without loss of generality, suppose $M>v$, and let $c\in[q_1,q_2]$ satisfy $f(c)=M$.

Consider the open interval $(v,M)$. Since $f(\mathbb{Q})$ is at most countable while $(v,M)$
is uncountable, we may choose a real number $y\in(v,M)\setminus f(\mathbb{Q})$.
By the Intermediate Value Theorem applied to $[q_1,c]$ and to $[c,q_2]$, there exist points
$x_1\in(q_1,c)$ and $x_2\in(c,q_2)$ with
\[
f(x_1)=y=f(x_2).
\]
Because $y\notin f(\mathbb{Q})$, neither $x_1$ nor $x_2$ can be rational.
Thus $x_1,x_2\in\mathbb{R}\setminus\mathbb{Q}$, and $x_1\neq x_2$.
This contradicts the injectivity of $f$ on $\mathbb{R}\setminus\mathbb{Q}$.

Therefore our assumption was false, and $f$ must be injective on $\mathbb{Q}$.
\end{proof}

\begin{exercise}
    Let $f: X \to Y,\: g: Y \to X$, if $\forall x \in X, g(f(x)) = x$, then $g$ is surjective 
    while $f$ is injective.
\end{exercise}

\begin{proof}
   $f$ is injective obviously, otherwise if $f(x_1) = f(x_2) = y$, then $h(x_1) = h(x_2) = g(y)$, which is contradict with
   $h$ is injective.

   $g$ is also surjective, if exists $x \in X$ and $x \notin g(Y)$. Then we got $h(x) \ne x$ since $x \notin g(Y)$
\end{proof}

\begin{exercise}
    Any vector space has a basis.
\end{exercise}

\begin{proof}
   Consider vector space $V$, which is not zero space. Let $E$ contains  
   all linearly independent subset of $V$. Consider $\subseteq$ as partial order.
   For any total order chain $C$, define 

   \[
        W = \bigcup_{U \in C} U
   \]

   We will show $W$ is linearly independent. Assume $x_1, x_2, \dots, x_n \in W$, and exists $c_1,c_2,\dots c_n$ such that

   \[
    c_1x_1 + c_2 x_2 + \dots + c_n x_n = 0
   \]

    Since $x_i \in U_i,\: 1 \le i \le n$, for some linearly independent subset $U_i$. 
    By $W$ is totally ordered, there exists $U'$ which is the 
    maximum element in $U_1,U_2,\dots ,U_n$. Since $x_i \in U', \: \forall 1 \le i \le n$, 
    and $U'$ is linearly independent, we got $c_1 = c_2 = \dots = c_n = 0$

\end{proof}

\begin{exercise}
    Let $f: \mathbb{R} \to \mathbb{R}$, consider set

    \[
        A = \{x \in \mathbb{R}: \lim_{x \to \infty} f(x) = \infty \}
    \]

    Then $A$ is countable
\end{exercise}

\begin{proof}
    Define

    \[
        E_N = A \cap {x \in \mathbb{R}: f(x) < N} \quad N = 0,1,2, \dots
    \]

    We will show each $E_N$ is isolated and hence countable.

    Assume exists $E_N$ which has at least one accumulation point. Pick $x' \in E_N$ and sequence $x_n \to x', x_n \ne x'$.
    Since $f$ has limit of $\infty$ at $x'$, we got $f(x_n) \to \infty$ here, which   
    contradicts $f(x_n) \le N$.

    After all, each $E_N$ is countable and $A$ is countable union of $E_N$ and hence countable.

\end{proof}