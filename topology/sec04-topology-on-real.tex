\section{Topology On Real}

\subsection{Topology on Real Line}

\begin{thm}
    Any open set under $\mathbb{R}$ is countable union
    of disjoint open intervals
\end{thm}

\begin{proof}
    If $U$ is open, for each $x \in U$ consider the collection $\mathcal{J}_x$ of all open
intervals $I$ such that $x \in I \subseteq U$. It is easy to check that the union of any family
of open intervals containing a point in common is again an open interval, and hence

\[
    J_x = \bigcup_{I \in \mathcal{J}_x} I
\]

is an open interval; it is the largest element of $\mathcal{J}_x$. If $x, y \in U$ then
either $J_x = J_y$ or $J_x \cap J_y = \varnothing$, for otherwise $J_x \cup J_y$ 
would be a larger open interval
than $J_x$. Thus if $\mathfrak{J} = { J_x : x \in U}$, the (distinct) members of $\mathfrak{J}$ are disjoint,
and 

\[
J = \bigcup_{J \in \mathfrak{J}} J
\]
For each $J \in \mathfrak{J}$, pick a rational number $f(J) \in J$. The map
$f : \mathfrak{J} \to \mathbb{R}$ thus defined is injective,  therefore $\mathfrak{J}$ is
countable.
\end{proof}

\begin{thm}
    Open sets in $\mathbb{R}^n$ is countable union of open balls.
\end{thm}

\begin{proof}
   Let $V \subseteq \mathbb{R}^n$ be an open sets. For any $x \in V$, there should exists $B(x, r_x) \subseteq V$. 
   Since $\mathbb{Q}$ is dense in $\mathbb{R}$, there should exists a rational point $q_x = (q_1,q_2,\dots,q_n) \in B(x, r_x/4)$.
   Put $r_{q(x)}$ be a rational in $(r_x /4 ,r_x/2)$, then $x \in B(q_x, r_{q(x)}) \subseteq B(x, r_x)$

   Thus we got:

   \[
    V = \bigcup_{x \in V} B(q_x ,r_{q(x)})
   \]

   The set

   \[
    \mathcal{I} = \{ B(q_x, r_{q(x)}): x \in V\}
   \]

   is countable, because we can made a injection from $h: \mathcal{I} \to \mathbb{Q} \times \mathbb{Q}$, which is 

   \[
    h\Big(B(q_x, r_{q(x)})\Big) = (q_x, r_{q(x)})
   \]
\end{proof}
